% Created 2017-05-04 Thu 06:30
% Intended LaTeX compiler: pdflatex
\documentclass[11pt]{article}
\usepackage[utf8]{inputenc}
\usepackage[T1]{fontenc}
\usepackage{graphicx}
\usepackage{grffile}
\usepackage{longtable}
\usepackage{wrapfig}
\usepackage{rotating}
\usepackage[normalem]{ulem}
\usepackage{amsmath}
\usepackage{textcomp}
\usepackage{amssymb}
\usepackage{capt-of}
\usepackage{hyperref}
\author{Chen Yangguang}
\date{\today}
\title{}
\hypersetup{
 pdfauthor={Chen Yangguang},
 pdftitle={},
 pdfkeywords={},
 pdfsubject={},
 pdfcreator={Emacs 25.1.2 (Org mode 9.0.5)}, 
 pdflang={English}}
\begin{document}

\tableofcontents

\#\# 自我能动
\begin{itemize}
\item 有机体既是它自己的因也是它自己的果,既是它自己固有的秩序和组织的因,也是其固有秩序和组织的果。
\item 每一个 \textbf{自我} 都是一个同义反复:自明、自指、以自己为中心并且自己创造自己。
\item 一个 \textbf{系统} ,就是任何一种能够自说自话的东西。
\item 自动控制三个阶段
\begin{itemize}
\item 由蒸汽机所引发的能量控制的第一阶段
\item 对物质的精确控制是第二阶段
\item 对信息本身的控制是第三阶段
\end{itemize}
\item 21 世纪的核心事件,是对 \textbf{信息的颠覆} 。
\item 自动化的历史,就是一条 \textbf{从人类控制到自动控制} 的单向通道。其结果就是从人类的自我到第二类自我的不可逆转的转移。而第二类自我是在我们控制之外的,是失控的。
\item 具有自我适应能力、像自己的目标进化、不受人类监管自行成长的机器,将会是下一个巨大的技术进步。要想获得有智能的控制,唯一的办法就是给机器自由。
\item 有些事必须先做,而且要按 \textbf{正确的次序} 去做。
\item \textbf{复杂的机器必定是逐步地,而且往往是间接地完善的。别指望通过一次华丽的组装就能完成整个工作正常的机械系统。你必须首先制作一个可运行的系统,再以此为平台研制你真正想完成的系统} (精辟呀!微信,Facebook也是一步步来的)。
\end{itemize}

\#\# 密封的瓶装生命

\begin{itemize}
\item 一天就可以摧毁的东西,要想建成它,可能会需要几年甚至几个世纪的时间。
\item 复杂性的开端根植于混沌之中。不过,如果复杂系统能够在一段时间的互相迁就之后获得共同的 \textbf{平衡} ,那么之后就再没有什么能够让它脱离轨道了。(是不是和习惯一样?一旦养成了自己固定的生活习惯,思维习惯,就会潜意识地贯彻下去。)
\item “封闭”意味着与流动隔绝。一个真正的封闭系统,是不会参与外部元素流动的; 换句话,它所有的循环都是自治的。
\item “系统”意味着相互 \textbf{连通} 。系统中的事物是相互纠结的,直接或者间接地连接到一个共同的命运。
\item \textbf{“存活”意味着惊喜} 。
\item 静态才是生态球的常态。
\item 生态球按比例放大后仍很完好。生态球越大,达到稳定所需时间就越长,破坏它也就越困难。只要处于正常状态,一个活系统的集体代谢过程就会扎下根,然后一直持续下去。
\end{itemize}

\#\# 生态技术玻璃球

\begin{itemize}
\item 生命经营的事业就是改造环境使其有益于生命。如果你能把生命聚拢成为一个群落,给它们充分的自由制造自己茁壮成长所需的条件,这个生物集合体就能一直生存下去,也没有必要知道它是怎样运转的。
\item 对于任何一个复杂、危险的项目来说,最理想的团队人数是8个人。超过8个人,会造成决策缓慢和耽搁;而少于8个人,突发事件或者疏忽大意就会变成严重的阻碍。
\item 细节是至关重要的东西。
\item 生命的任何领域都是由数不清的独立的回路编织而成的。
\end{itemize}

\#\# 合成生态系统

\begin{itemize}
\item 一旦你改变了生态系统, 并找到适合播种的的种子,以及必不可少的气候窗口,改变就开始了,而且这是不可逆转的。这个合成的生态系统持续运转下去并不需要人的存在,它不受干扰仍会保持下去。
\item 创造生物圈:
\begin{itemize}
\item 微生物做绝大部分的工作
\item 土壤是有机体, 它是活的。它会呼吸。
\item 创造【冗余】的食物网络
\item 逐步地增加多样性
\item 如果不能提供一种物理功能,就需要模拟一个类似的功能
\item 大气会传达整个系统的状态
\item 聆听系统,看看它要去哪里
\end{itemize}
\item 所有的封闭系统都是会被打开的,至少会出现泄露。(也就是: 不存在绝对封闭系统)
\end{itemize}
\#\# 冒出的生态圈
\begin{itemize}
\item 人体本身就是一个巨大的复杂系统。我们从自然中获得的施与是令人难以置信的。
\item 扰动是生态的必要催化剂。
\item 我们究竟可以切断多少的联系,还能保持一个物种的生存。
\item 命运始终具有讽刺意味。
\item 顺其自然就好。
\item 大气是极其重要的环境因素,大气产生生命,生命也可以产生大气。
\item 不要着急,不要在系统自组织的时候就急着催它孕育。你能给它的最重要的东西就是 \textbf{时间} 。
\item 所有复杂的共同进化系统都需要“ \textbf{冒出} ”。
\item 可以通过逐渐提高复杂性来重组大型系统;一旦一个系统达到了稳定水平,它就不会轻易地趋向于倒退,仿佛这个系统被新的复杂性带来的凝聚力所“吸引”。人类组织,比如团队和公司,也显示出 " \textbf{冒出} "的特征。
\item \textbf{生命就是技术} 。机器技术只不过是生命技术的临时替代品而已。随着我们对机器的改进,他们会变得更有机,更生命化,更近似生命, \textbf{因为生命是生物的最高技术} 。总有一天,机器和生物间的差别会很难区分。
\end{itemize}

\#\# 看不见的智能

\begin{itemize}
\item 最深刻的技术是那些看不见的技术,它们讲自己编织进日常生活的细枝末节之中,直到成为生活的一部分。书写的技术走下精英阶层,不断放低身段,从我们的注意力中淡出。
\item 电脑的胜利不但不会使世界非人化,反而会使环境更臣服于人类的愿望。我们创造的不是机器,而是讲我们所学所能融会贯通于其中的机械化环境。我们在将自己的生命延伸到周边环境中去。
\item 缺少了隐私技术的网络文化是无法兴旺的。
\item 哪里有生态系统,哪里就有精通本地事务的生命。
\item 人类渴求隐私,但事实上,我们的社会性胜过独立性。如果机器也像我们这样互相了解(甚至是一些很私密的事情),那么机器生态就是不可征服的。
\item 自然界没有垃圾问题,因为物尽其用。
\item 想大自然一样从事。
\item 机器是整体系统, 强调系统效率最大化。将自然环境的模式作为解决环境问题的模板。
\item 尽可以将所有能想到的废物都看作是潜在的原材料。任何在当下没有用的材料,都可以通过设计从源头将它消除。
\item 生态技术即使带不来令人震惊的利润,也会带来一定的成本收益。
\item 工业将无可避免地采用生物方式
\begin{itemize}
\item 它能用更少的材料造出更好的东西。
\item 自然是掌控复杂性的大师,在处理杂乱、反直观的网络方面给我们以无价的引导。未来的人造复杂系统为了能够运转,必然会有意识地注入有机原则。
\item 大自然是不为所动的,所以必须去适应她。
\item 自然界本身--基因和各种生命形式--与工业系统一样能够被工程化(或模式化)。
\item 生物学是一个必然--近于数学的必然,所有复杂性归向的必然。它是一个欧米茄(Omega)点。在天生和人缓慢的混合过程中,有机是一种显性性状,而机械是隐性性状。最终,获胜的总是生物逻辑。
\end{itemize}
\end{itemize}

\#\# 信息工厂
\begin{itemize}
\item 一个纯粹网络化的公司,应该具备这些特点
\begin{itemize}
\item 分布式
\item 去中心化
\item 协作
\item 可适应性
\end{itemize}
\item 在大多数情况下,真正的财富都是通过把某种流程置于集中控制之下而获得的。越大,效率越高。
\item 如果把一个任务拆分成若干块交给不同的公司来完成,若想保持质量的话,所需的交易成本要高于在一个公司内完成这项任务的成本。
\end{itemize}

\#\# 与错误打交道
\begin{itemize}
\item 软件的编制遵循三个中心化的关键步骤。
\begin{itemize}
\item 首先设计一个全景图
\item 然后用代码实现细节
\item 最后,在接近项目尾声的时候,将其作为交互的整体来进行测试。
\end{itemize}
\item 绝对不可能避免错误,但是可以避免错误成为缺陷。零缺陷设计的人物就是尽早发现错误,尽早改正错误。真正的改进在于尽早发现产生错误的原因,并尽早清除产生错误的原因。犯错的是人,处理错误的是系统。
\item 当你发现一个错误的时候,也就意味着还有另外一堆你没看见的错误在什么地方等着你。
\item 新生物学的解决之道是用一个个可以正常的单元来搭建程序性。
\item 一个好的程序员可以对任何一个已知的、规律的软件进行重写,巧妙地减少代码。但是,在创造性编程过程中,没有任何已经被完全理解的东西。
\end{itemize}

\#\# 联通所有的一切

\begin{itemize}
\item 新兴网络经济具备的特征
\begin{itemize}
\item 分布式核心
\item 适应性技术
\item 灵活制造
\item 批量化的定制
\item 工业生态学
\item 全球会计
\item 共同进化的消费者
\item 以知识为基础
\item 免费的带宽
\item 收益递增
\item 数字货币
\item 隐性经济
\end{itemize}
\item 这个时代里,最核心的行为就是把所有的东西都联结在一起。
\end{itemize}

\#\# 密码无政府状态 

\begin{itemize}
\item 要阻止信息的越界流动是一件毫无希望的事情。
\item 在信息时代,情报成为企业最主要的财富。
\end{itemize}

\#\# 数字货币

\begin{itemize}
\item 消费历史可以被用来构成一份精准且极具市场推广价值的档案。
\item 真正的数字现金是真钱,具有现金的私密性和电子的灵活性。
\item 安全性和私密性是矛盾的两面。
\item 信息流动到哪里,货币肯定也会跟随其后。货币是另一类信息,一种小型的控制方式。
\item 真正的数字现金,或者更准确地说,真正的数字现金所需要的经济机制,将会重新构造我们的经济、通讯以及知识。
\end{itemize}
\end{document}
